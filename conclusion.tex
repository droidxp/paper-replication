\section{Conclusions}

In this paper we reported the
results of two empirical studies that explore
techniques for Android malware identification.
The first study is a non-exact replication of a
previous research work~\cite{DBLP:conf/wcre/BaoLL18},
which investigates the Android mining sandbox
approach for malware identification. There,
Bao et al. report that more than 70\% of
the malwares in their dataset can be
detected by the sandboxes built from the
execution of five test case generation
tools (such as Monkey and DroidMate). Our replication
study revealed that this performance is only
achieved if we enable a static
analysis component from DroidFax that was supposed to only
instrument the Android \texttt{apk} files,
though that independently contributes to
building the sandboxes statically. As such,
the use of DroidFax leads to an
overestimation of the performance of the
mining sandbox approach supported by dynamic analysis.
Indeed, the execution of DroidFax alone enabled
us to generate a sandbox that can
identify 43.75\% of the malwares from 
their dataset. 

In the second study we investigated
a new approach based on taint analysis
for malware identification, which
leads to promising results. First,
the taint based static analysis approach
detected 60.42\% of the
malwares in the dataset. When combining
taint analysis with the
mining sandbox approach, we were
able to identify 82.29\% of the
malwares in the dataset. These results
have implications for both researchers and
practitioners. First, we review the literature
showing, for the first time, empirical evidence
that the mining sandbox approach benefits
from using both dynamic and static analysis. Second,
practitioners can improve malware identification using
a combination of 
the mining sandbox approach with taint analysis.
Nonetheless, both the mining sandbox approach and
taint analysis present limitations. In particular,
we are not able to identify a malware that
uses the same set of calls to sensitive APIs of the
benign version of an app, using
the mining sandbox approach. Similarly, we are not
able to identify a malware that presents the same
paths from sources to sinks of the corresponding benign version of
an app, using the taint analysis approach. 
To mitigate these limitations, we envision the use of other approaches---such as machine learning
algorithms to classify changes in non-code assets (e.g., Android
manifest files) and symbolic execution to differentiate
malicious calls or source-sink paths.
