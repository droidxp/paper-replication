\begin{abstract}
The popularization of Android platform and the growing number of Android apps that manage sensitive data, brought several security threats. Over the past years, researchers and practitioners have investigated techniques addressing Android's security issues, including techniques that leverage dynamic analysis to mine Android Sandboxes. Mining sandboxes through test generation tools is not a new strategy, and has been investigated before, however a potential combination with static analysis has not been sufficiently explored yet.

In this paper we investigate whether or not the use of static analysis might complement and increase the performance of dynamic analysis tools for mining Android sandboxes. Therefore, we first conducted a non-exact replication of a previous study \cite{DBLP:conf/wcre/BaoLL18} that compares dynamic test case generation tools for mining sandboxes, isolating all possible effects of the static analysis, that could leverage the results. We then conducted a new study to investigate the performance of tainted analysis algorithms for mining sandboxes.

Our results demonstrate that static analysis has an important contribution to the sandbox mining process, and its combination with dynamic analysis has the potential to improve malware detection techniques.


\end{abstract}

\begin{IEEEkeywords}
Malware Detection,
Mining Sandboxes,
Software Security,
Android Platform,
Empirical Studies,
Static Analysis and Dynamic Analysis
\end{IEEEkeywords}