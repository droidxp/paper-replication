\begin{abstract}
The popularization of Android platform and the growth number of Android apps that manage sensitive data, brought several security threats. During the last years researchers and practitioners have investigated techniques addressing security Android issues, including techniques that leverage dynamic analysis to mine Android Sandboxes. Mining sandboxes through test generation tool is not new, and has been investigated before, however a potential combination with static analysis has not been sufficiently explored.

In this paper we investigate whether or not the use of static analysis might complement and increase the performance of dynamic analysis tools for mining Android sandboxes. To this end, we first conducted a non-exact replication of a previous study \cite{DBLP:conf/wcre/BaoLL18} that compare the results of test case generation for mining sandboxes, isolating all possible effects of the static analysis on the results, that could leverage the original results. We then conducted a new study to investigate the performance of tainted analysis algorithms for mining sandboxes.

Our results demonstrate that analysis static has an important contribution to the mining sandboxes process, and his combination with dynamic analysis has the potential to improve malware detection techniques.


\end{abstract}

\begin{IEEEkeywords}
Malware Detection,
Mining Sandboxes,
Software Security,
Android Platform,
Empirical Studies,
Static Analysis and Dynamic Analysis
\end{IEEEkeywords}