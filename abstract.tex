\begin{abstract}
%  \emph{Context.}
%  The popularization of the Android platform and the growing number of Android apps that manage sensitive data turned the Android ecosystem into an attractive target for malicious software. For this reason, researchers and practitioners have investigated new approaches to address Android's security issues, including techniques that leverage dynamic analysis to mine Android Sandboxes. 
%  \emph{Problem.}
  The use of sandboxes is an effective technique for malware analysis. However, although the use of dynamic analysis for mining Android sandboxes has been investigated before, little is known about the potential benefits of combining static and dynamic analysis for mining Android sandboxes.
Accordingly, in this paper we present the results of two studies that investigate whether or not static analysis might complement and increase the performance of dynamic analysis tools for mining Android sandboxes. %\emph{Method.} 
In the first study we conduct a non-exact replication of the Bao et al. work, a previous study that compares the performance of test case generation tools for mining Android sandboxes. Differently from the original work, here we isolate the effect of the static analysis tool (DroidFax) they used to instrument the Android apps in their experiments. This decision was motivated by the fact that DroidFax could have influenced the efficacy of the dynamic analysis positively---through the execution of specific static analysis algorithms. In our second study, we carried out a new experiment to investigate the efficacy of tainted analysis algorithms for mining Android sandboxes. To this end, we executed the FlowDroid tool to mine the source-sink flows from the benign/malign pairs of Android apps used in the Bao et al. study. %\emph{Results.}
The first study reveals that DroidFax alone (static analysis) can detect 43.75\% of the malwares in the dataset of the Bao et al. work, increasing the performance of the dynamic analysis tools substantially. The results of the second study show that (static) tainted analysis is also practical for mining sandboxes, with a performance similar to that reached by dynamic analysis tools.
\end{abstract}

\begin{IEEEkeywords}
Malware Detection,
Mining Sandboxes,
Software Security,
Android Platform,
Empirical Studies,
Static Analysis and Dynamic Analysis
\end{IEEEkeywords}
