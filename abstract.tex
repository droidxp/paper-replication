\begin{abstract}
Due to the popularization of Android and the full range of applications (apps) targeting this platform, many security issues have emerged, attracting researchers and practitioners' attention. As such, many techniques for addressing security Android issues appeared, including approaches for mining sandboxes. Previous research studies have compared Android test case generation tools for this specific goal. Our research aims to explore new techniques for mining sandboxes, especially we are interested in understanding the limits of both static and dynamic analysis in this process. Although the use of test generation tools for mining sandboxes has been explored before, the potential to combine static analysis and dynamic analysis has not been sufficiently investigated yet. That is, in this paper, we will investigate the hypothesis that combining static and dynamic analysis techniques increases the process of mining Android sandboxes. To understand the bound of this combination, we conducted a non-exact replication of a previous study to detect what is the role of each of these analyses in malware detection using mining sandboxes. Like the original studies, we used DroidFax, a tool that instruments Android apps and collects relevant information about their execution (using the test case generation tools). DroidFax also executes static analysis that collects code coverage information and a set of sensitive APIs used by a given app during test execution. Our results demonstrate that analysis static has an important contribution to the mining sandboxes process, and his combination with dynamic analysis has the potential to improve malware detection techniques.

\end{abstract}

\begin{IEEEkeywords}
Malware Detection,
Mining Sandboxes,
Software Security,
Android Platform,
Empirical Studies,
Static Analysis and Dynamic Analysis
\end{IEEEkeywords}