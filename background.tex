\section{Background and related works}

A sandbox is an isolated environment on an electronic device within which applications cannot affect other programs outside its boundaries. It enables testing and execution of unsafe or untested code, possible malware, without worrying about the integrity of the electronic device. It also guarantee an on-biased testing environment, ensuring the same pre-configured beginning status of the environment.

In this sandbox context, the Android Mining Sandbox concept emerges as a technique that consists of extract rules from android app, and use these rules to ensure system security. The technique performs two steps. First, rules are mined and will compose the sandbox, through test generator tools. These tools explore program behavior and monitor access to sensitive APIs. In a second stage, it is taken into account that resources not accessed, or accessed differently, at first stage, must be denied access. So, if a malicious app requires access to resources, different from what was previously mined, the sandbox will prohibit this access.

The idea of mine automatically software resources or components, to inferring behavior is not new, and has been discussed before. For instance, Whaley et al. \cite{whaley2002automatic} combine dynamic and static analysis to API mining and so extract interface from software components. Ammons et al \cite{ammons2002mining} propose a machine learning
approach, called specification mining, to discovering some temporal and data-dependence relationships that a program follows when interacting with an API or abstract data type.

Regarding test generating tools used to mining Sandboxes, Jamrozik et al \cite{jamrozik2016droidmate} proposed DroidMate, a test generation tools that implements a pseudo-random GUI exploration strategy, and was the first approach to leverage test generation to extract sandbox rules from apps. Li e tal. proposed in \cite{li2017droidbot}, DroidBot, a test generator tool that explores sensitive resources access by apps, following a model-based exploration strategy. In this paper, the authors presented comparison between DroidBot and Monkey \cite{Monkey} in a proof of concept example of using DroidBot in malware analysis. From the same authors, another test generator tool for Android, described as Humanoid \cite{humanoid-paper}, is a Droidbot evolution and presents a proposal that can generate humans like tests inputs, using deep learning.

Previous research studies have investigated the effectiveness of mining sandboxes, using automated testing tools, to compare the performance of each tool to mine sandboxes. For instance. For example, L Bao et al. \cite{bao2018mining} had provided an important comparative study between five test generation tools in finding malware using mining sandboxes techniques: DroidMate \cite{jamrozik2016droidmate}, Monkey \cite{Monkey}, GUIRipper \cite{amalfitano2012using}, Puma \cite{hao2014puma}, and Droidbot \cite{li2017droidbot}. His first experiment used a small set of apps (10 pairs B/M) to be run for different periods to compare how much time the tools needed to achieve the maximum coverage possible and to identify the sensitive APIs accessed. This study showed that the coverage and accuracy of the testing tool have a negative correlation. That is, improving the performance of code coverage might not directly contribute to the performance of detecting malware. The second experiment used a set of 102 pairs of apps that were executed for just one minute. The goal of this experiment was to compare a bigger number of apps and the number of sensitive APIs accessed by each tool in a small time.