\section{Background and related works}

A sandbox is an isolated environment on an electronic device within which applications cannot affect other programs outside its boundaries. It enables testing and execution of unsafe or untested code, possible malware, without worrying about the integrity of the electronic device. It also guarantee an on-biased testing environment, ensuring the same pre-configured beginning status of the environment.

In this sandbox context, the Android Mining Sandbox concept emerges as a technique that consists of extract rules from android app, and use these rules to ensure system security. The technique performs two steps. First, rules are mined and will compose the sandbox, through test generator tools. These tools explore program behavior and monitor access to sensitive APIs. In a second stage, it is taken into account that resources not accessed, or accessed differently, at first stage, must be denied access. So, if a malicious app requires access to resources, different from what was previously mined, the sandbox will prohibit this access.

The idea of mine automatically software resources or components, to inferring behavior is not new, and has been discussed before. For instance, Whaley et al. \cite{whaley2002automatic} combine dynamic and static analysis to API mining and so extract interface from software components. Ammons et al \cite{ammons2002mining} propose a machine learning
approach, called specification mining, to discovering some temporal and data-dependence relationships that a program follows when interacting with an API or abstract data type.

Regarding test generating tools used to mining Sandboxes, Jamrozik et al \cite{jamrozik2016droidmate} proposed DroidMate, a test generation tools that implements a pseudo-random GUI exploration strategy, and was the first approach to leverage test generation to extract sandbox rules from apps. Li e tal. proposed in \cite{li2017droidbot}, DroidBot, a test generator tool that explores sensitive resources access by apps, following a model-based exploration strategy. In this paper, the authors presented comparison between DroidBot and Monkey \cite{Monkey} in a proof of concept example of using DroidBot in malware analysis. From the same authors, another test generator tool for Android, described as Humanoid \cite{humanoid-paper}, is a Droidbot evolution and presents a proposal that can generate humans like tests inputs, using deep learning.

L Bao et al. \cite{bao2018mining} had provided an important comparative study between five test generation tools in finding malware using mining sandboxes techniques: DroidMate \cite{jamrozik2016droidmate}, Monkey \cite{Monkey}, GUIRipper \cite{amalfitano2012using}, Puma \cite{hao2014puma}, and Droidbot \cite{li2017droidbot}. This study indicates that these tools was efficient in identifying at most 70\% of the malware in a specific dataset and also reports that, even after combining all test generator tools, it was possible to detect 80\% of malicious behavior. However, this study did not focus on the possible interference of static analysis in the final result, since this study used a instrumentation tool \cite{cai2017droidfax}, that also performs a static analysis of the apps.