\section{Background and related works}

A sandbox is an isolated environment on an electronic device within which applications cannot affect other programs outside its boundaries, like the file system, the network or other device data. It enables testing and execution of unsafe or untested code, possible malware, without worrying about the integrity of the electronic device that runs the application. 

%It also guarantees an unbiased testing environment, ensuring the same pre-configured beginning status of the environment \kn{I dont understand what a pre-configured beginning status is}.

The Android Mining Sandbox concept is a sandboxing technique that consists of mining rules from an android app, and use these rules to ensure system security. The sandboxing technique comprises of two steps. First, rules are miner and will compose the sandbox,
%\kn{I thought sandbox was an environment, now it is comprised of rules?}%
through test generator tools. These tools explore program behavior and monitor access to sensitive APIs. The second stage ensures that resources not accessed, or accessed differently, at first stage, are denied access. So, if a malicious app requires access to resources, different from what was previously mined, the sandbox will prohibit this access.

Automatically mining software resources or components, to infer behavior is not new, and has been discussed before. For instance, Whaley et al. \cite{DBLP:conf/issta/WhaleyML02} combine dynamic and static analysis for API mining and so extract interface from software components. Ammons et al \cite{DBLP:conf/popl/AmmonsBL02} propose a machine learning
approach, called specification mining, to discover temporal and data-dependence relationships that a program follows when interacting with an API or abstract data type.

Regarding test generating tools used for mining Sandboxes, Jamrozik et al \cite{DBLP:conf/icse/JamrozikZ16} proposed DroidMate, a test generation tool that implements a pseudo-random GUI exploration strategy, and was the first approach to leverage test generation to extract sandbox rules from apps. Li e tal.~\cite{DBLP:conf/icse/LiYGC17} proposed DroidBot, a test generator tool that explores sensitive resources access by apps, following a model-based exploration strategy. In their work, the authors present a comparison between DroidBot and Monkey~\cite{Monkey} regarding malware analysis, and showed that Droidbot is able to trigger an amount of sensitive behaviors, like data leaks and file accesses, higher than Monkey. From the same authors, another test generator tool for Android, described as Humanoid \cite{DBLP:conf/kbse/LiY0C19}, is a Droidbot evolution and presents a proposal that can generate humans like tests inputs, using deep learning.

%~\kn{What is being compared with what. Is droidbot compared with monkey. If yes, then what is the proof  of concept example of using Droidbot? Something we have built. Or is the comparison between Droidbot and Monkey as an example of using droidbot?}. 



L Bao et al~.\cite{DBLP:conf/wcre/BaoLL18} had provided an important comparative study between five test generation tools that find malware by mining sandboxes techniques: DroidMate, Monkey, GUIRipper \cite{DBLP:conf/kbse/AmalfitanoFTCM12}, Puma \cite{DBLP:conf/mobisys/Hao0NHG14}, and Droidbot. This study indicates that these tools was efficient in identifying at most 70\% of the malware in a specific dataset and also reports that after combining all test generator tools, it was possible to detect 75.49\% of malicious behavior explored ($77$ among $102$). However, this study did not focus on the possible interference of static analysis in the final result, since this study used a instrumentation tool \cite{DBLP:conf/icsm/CaiR17a}, that also performs a static analysis of the apps 

Android applications contain within themselves the risk that sensitive data, such as credit card details, device information can be leaked into public sinks like the internet. Taint analysis is a special type of static analysis that enables tracking of sensitive data within programs~\cite{boddenesec}. Wei et al.~\cite{weiissta} propose a scalable taint analysis for Android applications that applies traditional taint analysis techniques with targeted optimizations specific to the android operating system. Flowdroid~\cite{DBLP:conf/pldi/ArztRFBBKTOM14} improves on precision of traditional approaches by including context and flow sensitivity. A significant issue with taint analysis is the cost of the tool itself hampering the performance. FastDroid~\cite{ZHANG2021102161} mitigates this issue by introducing an intermediate light-weight abstraction to perform the analysis. 

%Although our study does not incorporate any taint analysis, one can imagine that applying taint analysis as an extra static analysis can drastically improve the precision of the dynamic analysis tools. 
