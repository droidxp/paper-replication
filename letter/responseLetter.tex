\documentclass[12pt,english]{scrartcl}

\usepackage{scrletter}
\usepackage[utf8]{inputenc}
\usepackage[T1]{fontenc}
\usepackage[]{babel}
\usepackage[numbers]{natbib}

% \usepackage{natbib}
% \bibliographystyle{unsrtnat}

%% \newkomavar{fromplace}
%% \setkomavar{fromname}{Francisco Handrick da Costa (on behalf of the authors of the paper}
%% \setkomavar{fromaddress}{Computer Science Department \\ University of Bras\'{i}lia}
%% \setkomavar{fromplace}{Brazil}




%% \firsthead{%
%% \begin{tabular}{p{0.7\textwidth}p{0.3\textwidth}}
%% & \usekomavar{fromname} \newline \usekomavar{fromaddress}  \newline \usekomavar{fromplace} 
%% \end{tabular}
%% }



% \documentclass{letter}

% \usepackage{natbib}
\usepackage{hyperref}
\usepackage{xcolor}

\DeclareOldFontCommand{\bf}{\normalfont\bfseries}{\mathbf}


% \signature{Francisco Handrick Tomaz da Costa (on behalf of the authors of the paper)}

%\address{Computer Science Department \\ University of Bras\'{i}lia \\ Brazil \\\\ Software Technology Group
% \\ Technical University of Darmstadt \\ Germany}


\begin{document}
\setkomavar{fromname}{Francisco Handrick da Costa}
\setkomavar{fromaddress}{Computer Science Department \\ University of Bras\'{i}lia \\ Brazil}

\begin{letter}{Dear JSS Editors,}
  
\opening{The comments of the reviewers have been very useful. They made us aware that some important points we were trying to convey remained obscure. We have corrected infelicities of presentation and all errors that the reviewers brought to our attention. The text in {\color{blue}blue} indicates how we have dealt with each individual comment.}


{\bf Reviewer 1}

\begin{itemize}
\item In the first place, what I missed is an overview of the empirical studies conducted and the relation between 
first and second experiment. I would recommend to include, before Sections 3.2 and 3.3, a section 
presenting the overall research methodology and the connections between the studies - in this sense, 
a figure reporting the various steps would be appreciated; Figure 2 only summarizes the design of the first study.

\vspace{0.2cm}

\item In the second place, I would recommend to include more details on the experiments conducted. This is especially 
true for the second study, where Section 3.3 is a bit shallow. The three steps of the methodology are indeed shortly 
described without providing enough details, hence possibly precluding verifiability and replicability. In addition, the 
methodology is not linked to the research questions: I would recommend to restructure Section 3 in order to have 
data extraction processes and data analysis procedures for each research question.


\vspace{0.2cm}

{\color{blue}{\bf Answer.} Thanks for raising these issues. We reviewed Section 3 in full,
to detail our approach better. In particular, we expanded the methodology of our
second study (Section 3.3) and highlighted our data collection and analysis procedures.
Regarding verifiability and replicability, we improved the documentation of our
packages and tools (see the next comment).}  

\vspace{0.2cm}

\item The proposed DroidXP benchmark seems not to be available, which is a pity since it might be very useful for other 
researchers. I believe the paper may have a stronger impact if the tool would be available. The same is true for the 
data and scripts used for the experiments. I would appreciate if the paper may have an online repository making all 
data and tools available.


\vspace{0.2cm}

{\color{blue}{\bf Answer.} Thanks for this comment. All tools, datasets, and scripts we used in our research are available on-line. The following repositories present detailed instructions about these assets: 

  \begin{itemize}
    \item \href{https://github.com/droidxp/benchmark}{DroidXP GitHub repository}.
    \item \href{https://htmlpreview.github.io/?https://github.com/droidxp/paper-replication-package/blob/master/replication.html}{Datasets and scripts}
  \end{itemize}
  
  \vspace{0.2cm}
  
  We also included references to these repositories in the first section (Introduction) of the current version of
  our paper.}

\vspace{0.2cm}

\item Another point to improve is connected to the implications of the study. While the paper discusses in nice way the 
results achieved, it does not provide a clear view of how those results can be potentially useful for researchers 
working on Android sandboxes as well as what are the next steps that the research community should pursue to 
improve the detection of malware.


\vspace{0.2cm}

{\color{blue}{\bf Answer.} You are right. To mitigate this issue, we created a new Section 5 (Implications), that present how our results can be useful for both researchers and practitioner.}


\vspace{0.2cm}

\item Last but not least, the threats to validity section should be improved, by commenting on conclusion and construct validity.


\vspace{0.2cm}

{\color{blue}{\bf Answer.} Thanks for this observation. We reviewed the Section 6 (Threats) to detail better internal and external validity. Concerning construct validity, we also add it at this Section.}


\end{itemize}

{\bf Reviewer 2}

\begin{itemize}

\item The abstract is not clear. First, the context if not well addressed.
  For example, why ``mining Android sandboxes'' is important? I had to read
  several related reference at the beginning to understand it. Second, it is not common practice to 
include reference in abstract. However, I reserve my opinion since I know it may be preferred by some researchers.


\vspace{0.2cm}

{\color{blue}{\bf Answer.} Thanks for raising these questions. We reviewed the abstract
  to better explain the relevance of the mining sandbox approach. Literature reports
  show that the mining sandbox approach is effective  not only to prevent malicious
  calls to sensitive APIs but also to classify a version of an app as a malware.}

\vspace{0.2cm}

\item In Figure 2, you used ``Malicious'', but in most other places, you used ``Malign''.
  Please unify your choice of terminology.


\vspace{0.2cm}

{\color{blue}{\bf Answer.} Thanks for this observation. We unified our terminology throughout the current
  version of the paper.}

\vspace{0.2cm}

\item In 3.3, you mentioned ``we investigate whether or not a more advanced static analysis approach is promising for mining 
Android sandboxes. To this end, we leverage the FlowDroid taint analysis algorithms...''. True, FlowDroid has its edge 
in privacy leakage detection, but what's your justification for saying FlowDroid is more advanced in terms of static analysis?


\vspace{0.2cm}

{\color{blue}{\bf Answer.} Thanks for raising this question. Here we were intended to say that FlowDroid implements a more advanced static analysis approach than DroidFax. We agree that the sentence is ambiguous, and we fixed that in the new version of the paper (both in Introduction and in the first paragraph
  of Section 3.3).}

\vspace{0.2cm}

\item By definition, mining sandboxes means a technique to confine an application to resources accessed during automatic 
testing~\cite{jamrozikZ16}. But in 3.3, you are comparing the source-sink pairs between that of a benign app and its malign twin. It is already not ``Mining Sandboxes''.

\vspace{0.2cm}

{\color{blue}{\bf Answer.} We totally agree. We changed the title of Section 3.3 and fixed this wrong mischaracterization related to our taint analysis usage, which is used not for mining sandboxes but rather as an alternative approach for identifying malware. Fixing this issue required modifications not only in Section 3.3 but also in many other parts of the paper (including the paper Title, Abstract, Introduction, and Results). Many thanks for raising this issue.}


\vspace{0.2cm}

\item Also, FlowDroid requires as input the sources and sinks APIs. But you mine ``all sources and sinks 
paths''. Do you mean all APIs? FlowDroid will never finish running. 

\vspace{0.2cm}

{\color{blue}{\bf Answer.} You are right: we actually use a curate set of sensitive sources and sinks
  that is available in the FlowDroid repository.
  We changed Section 3.3 to clarify this point. Thanks once again.
  We hope that the current version of our paper fixes this problem.}

\vspace{0.2cm}
  
\item Assume a set of sensitive sources and sinks are selected, 
  FlowDroid is already good in suggesting which one is malign by revealing info flows. What's the point of
  applying~\cite{jamrozikZ16}'s setting?


\vspace{0.2cm}

{\color{blue}{\bf Answer.} Our approach in the second study aims to find divergences in the source-sink
  flows when considering two executions of FlowDroid: one execution using the benign version and
  one execution using the malicious version. The source-sink paths collected
  in the first execution of FlowDroid (benign version)
  {\bf are not considered suspicious}. Only the paths in the second execution, which do not appear in the first execution, are considered suspect and are used to point out the malware. We changed Section 3.3 to
  detail our approach better.}

\vspace{0.2cm}

\item The significance of this work is questionable. The first study is a mere replication of the existing
  work~\cite{DBLP:conf/wcre/BaoLL18}, except, disabling 
part of the framework. The finding is that DroidFax is important part for DroidXP. Well, your proved literature's correctness. 
It would be more impressive the other way round.


\vspace{0.2cm}

{\color{blue}{\bf Answer.} The relevance of replication in software engineer
  has gained much more attention in recent years~\cite{role-of-replication,shepperd-role-replication},
  not only to validate past results~\cite{msr-replication} but also to generate new insights
  from previous research results~\cite{kohl:icse-2020} or even refute past conclusions~\cite{toplas-replication}.
  Indeed, we carried out an external, non-exact replication
  of~\cite{DBLP:conf/wcre/BaoLL18}, and the results of our study
  shed light that the efficacy of the dynamic analysis tools is
  actually different from the previous report~\cite{DBLP:conf/wcre/BaoLL18}. That is,
  although we agree with the perspective of
  Shullet et al.~\cite{role-of-replication},
  who argue that an exact replication that achieves the same conclusions of an original work
  is still valuable to generalize the results of a research, here we go beyond that,
  and give evidence that the efficacy of the dynamic analysis tools in the previous research
  work had been overestimated. We review the introduction section to clarify this point.}

\vspace{0.2cm}

\item I find what's written in Finding 2 is not exciting. It is intuitive based on the results
  of~\cite{jamrozikZ16} to know that a comprehensive 
comparison combining multiple dynamic tools would improve the performance.


\vspace{0.2cm}

{\color{blue}{\bf Answer.}\emph{Finding 2} summarizes that we achieve the best performance on mining sandboxes when combining Humanoid, Monkey, and DroidBot. Adding DroidMate to this configuration does not improve the overall performance. This result is novel and different from the previous research. We have made a slight change in this particular box to stress this point.}

\vspace{0.2cm}

\item Listings 1---7 illustrate more details of representative malware in the dataset. But what is the take-away message? What 
conclusions and/or inspirations can be drawn from such listings?


\vspace{0.2cm}

{\color{blue}{\bf Answer.}The goal of the listings is twofold. First, to explain the characteristics of the malwares for which the mining sandbox approach could correctly identify the malicious behavior. Second, to highlight a particular kind of malwares for which the mining sandbox approach is not practical (last malware example). For the first class of malware, the malicious version of the apps frequently includes changes in the Android manifest file and also modifies the bytecode to include additional method calls that might share sensitive information. The listings show concrete examples of the code after a re-engineering effort, which might help readers understand why the mining sandbox approach works for those cases. Similarly, the last instance of malware only changes the manifest file. Since it does not include any additional method call, the mining sandbox approach could not point out the malicious version of the app as malware. We changed the last paragraphs of Section 4.1 to justify these listings better.
}

\vspace{0.2cm}

\item The title itself is inaccurate. You removed DroidFax from the replicated work~\cite{DBLP:conf/wcre/BaoLL18}.
  But it is not the only static analysis part. Actually, many dynamic testing tools use
  static analysis to better guide their testing strategy.


\vspace{0.2cm}

{\color{blue}{\bf Answer.} We agree. Nonetheless, DroidFax is not part of the dynamic analysis tools. According to Bao et al.~\cite{jamrozikZ16}, their goal on using DroidFax was to instrument the APKs with the intent of collecting calls to sensitive APIs, while executing the dynamic analysis tools. However, the use of DroidFax in their study actually lead to an increase in the performance of dynamic analysis tools for mining sandboxes. We did not change any internal component of the dynamic analysis tools, which indeed might include static analysis components. We reviewed the title to mitigate this problem.}


\end{itemize}

{\bf Reviewer 3}

\begin{itemize}

\item It is not clear how this paper can be considered an extension of the DroidXP paper presented in SCAM 2020.


\vspace{0.2cm}

{\color{blue}{\bf Answer.} In the introduction, we rephrase paragraphs 4 to better clarify how our paper is an extension of DroidXP paper from SCAM 2020.}

\vspace{0.2cm}

\item Introduction, para 3 is repeating para 2


\vspace{0.2cm}

{\color{blue}{\bf Answer.} Paragraphs 2 and 3 have indeed similar content. We reviewed Section 1 (Introduction) and removed paragraph 3.}

\vspace{0.2cm}

\item ``The authors state that dynamic analysis exceeds static analysis on mining Android sandboxes because the
analysis must often assume that additional behaviors are possible than actually would be.''
Please explain


\vspace{0.2cm}

{\color{blue}{\bf Answer.} We explained better this sentence at introduction, paragraph 2. Thanks for raising this issue.}

\vspace{0.2cm}

\item ``We presented and evaluated DroidXP in a conference paper [11]''
What is DroidXP? It should be introduced to a reader who is not familiar.


\vspace{0.2cm}

{\color{blue}{\bf Answer.} We agree. To mitigate this issue, we introduced better DroidXP at introduction, paragraph 4.}

\vspace{0.2cm}

\item Google play - Google Play
* java - Java / android - Android
* ``Another approach to rooting''
What is rooting?


\vspace{0.2cm}

{\color{blue}{\bf Answer.} Sorry about these typos. We corrected all of them. Thank one more time. With respect to ``rooting'': It's a process performed on Android devices like smartphones or tablets, which allows users of the Android mobile operating system, to execute privileged commands that are typically unavailable at its default configuration. The term comes from the fact that Android system is based on Linux, and this system has a (superuser), called root, which has access to administrative permissions as on Linux. We insert a footnote with a link that better explain this term at page 5.\newline
(https://en.wikipedia.org/wiki/Rooting(Android))
}

\vspace{0.2cm}

\item Please rephrase:
``Those are kind of attack that usually tackled with the Android Mining Sandbox''
``For instance, Whaley et al. [18] combine dynamic and static analysis for API mining and so extract interface from software components.''


\vspace{0.2cm}

{\color{blue}{\bf Answer.} With respect to the first sentence, we decided to remove it, since we realized that it was misleading. The second sentence we rephrase like : ``For instance Whaley et al. [18] combine dynamic and static analysis for API mining and so infer program behavior based on an example use of a Java class.''}

\vspace{0.2cm}

\item What are sensitive behaviors? ``showed that DroidBot is able to trigger an amount of sensitive behaviors''


\vspace{0.2cm}

{\color{blue}{\bf Answer.} Sensitive behaviors at Android context occurs when an Android app functionality can result in accessing or leaking of users sensitive data. Example of sensitive behaviors: File accesses and network access. We improve the text with this definition at Section 2 (Background and related works), paragraph 7.}

\vspace{0.2cm}

\item Please rephrase:
``From the same authors, another test generator tool for Android, described as Humanoid [23], is a DroidBot evolution and
presents a proposal that can generate humans like tests inputs, using deep learning.''


\vspace{0.2cm}

{\color{blue}{\bf Answer.} We rephrase the text at Section 2 (Background and related works) paragraph 7, to better explain Humanoid test generation tool.}

\vspace{0.2cm}

\item Briefly explain the taint analysis
``Taint analysis is a special type of static analysis that enables tracking of sensitive data within programs''


\vspace{0.2cm}

{\color{blue}{\bf Answer.} We agree that taint analysis explanation was very briefly. We reviewed our Section 2 (Background and related works), to better explain taint analysis from paragraph 9. We explained better the concept of ``source'' and ``sink'', and present how the analysis identifies sensitive information leakage, detecting taint flow between “source” and “sink”. To improve understanding, we also present a simple Listing (Listing 1), that present a simple data leakage example through data flows. }


\vspace{0.2cm}

\item Please explain:
``FastDroid [28] mitigates this issue by introducing an intermediate light-weight abstraction to perform the analysis.''


\vspace{0.2cm}

{\color{blue}{\bf Answer.} To address this issue, we explained better how FastDroid improves the taint analysis efficiency at the last paragraph of Section 2 (Background and related works)}

\vspace{0.2cm}

\item Section 3.1, the help text is not adding much value

\vspace{0.2cm}

{\color{blue}{\bf Answer.} Thanks for raising this point. The section 3.1 presented our Benchmark tool DroidXP. The tool description was important for the paper since it was helpful for our first study conduction. DroidXP helped us to easily integrate different test generation tools, and understand the effect of the static analysis component at each tool. It was useful for us and can be very useful for other researchers. We
reviewed the first paragraph of section 3.1 to clarify the importance of DroidXP for our study conduction.}


\vspace{0.2cm}

\item Please rephrase: ``After that, we executed the analysis again, though disabling the DroidFax static analysis algorithm''

\vspace{0.2cm}

{\color{blue}{\bf Answer.} We reviewed the Section 3 (Study Settings) in full and so, we better explain the methodology. With this update, we removed this sentence.}

\vspace{0.2cm}

\item Section 4.1 can be written more concisely.
* It is not clear what is the value of the malware discussion between pages 13 and 15. It may be removed.

\vspace{0.2cm}

{\color{blue}{\bf Answer.} Thanks for raising this issue. Actually our purpose was to present the characteristics of malware that have or have not been identified using the mining sandbox approach. Our goal is to provide a lower-level intuition about the classes of malware, the mining sandbox approach is good enough, and which ones it fails. We rephrased the section 4.1 at this point, to clarify this issue about malware discussion.}

\vspace{0.2cm}

\item There are several other typos and grammatical errors.


\vspace{0.2cm}

{\color{blue}{\bf Answer.} Sorry about these typos and grammar. We revised all the text, and corrected several mistakes that went unnoticed.}





\end{itemize}



{\bf Meta review}

We want to thank all comments we received. Based on this feedback, we conducted additional research activities and improved the paper in several directions.

\closing{Yours Faithfully,}


\bibliographystyle{abbrvnat}
\bibliography{references}

\end{letter}
\end{document}
