\documentclass{letter}

\usepackage{hyperref}
\usepackage{xcolor}

\signature{Francisco Handrick Tomaz da Costa (on behalf of the authors of the paper)}

\address{Computer Science Department \\ University of Bras\'{i}lia \\ Brazil \\\\ Software Technology Group
 \\ Technical University of Darmstadt \\ Germany}




\begin{document}

\begin{letter}{JSS Editors,}
\opening{The comments of the reviewers have been very useful. They made us aware that some important points we were trying to convey remained obscure. We have corrected infelicities of presentation and all errors that the reviewers brought to our attention. The text in {\color{blue}blue} indicates how we have dealt with each individual comment.}


{\bf Reviewer 1}

\begin{itemize}
\item In the first place, what I missed is an overview of the empirical studies conducted and the relation between 
first and second experiment. I would recommend to include, before Sections 3.2 and 3.3, a section 
presenting the overall research methodology and the connections between the studies - in this sense, 
a figure reporting the various steps would be appreciated; Figure 2 only summarizes the design of the first study.


\vspace{0.2cm}
  
{\color{blue}{\bf Answer.} answer}

\vspace{0.2cm}

\item In the second place, I would recommend to include more details on the experiments conducted. This is especially 
true for the second study, where Section 3.3 is a bit shallow. The three steps of the methodology are indeed shortly 
described without providing enough details, hence possibly precluding verifiability and replicability. In addition, the 
methodology is not linked to the research questions: I would recommend to restructure Section 3 in order to have 
data extraction processes and data analysis procedures for each research question.


\vspace{0.2cm}

{\color{blue}{\bf Answer.} answer}  

\vspace{0.2cm}

\item The proposed DroidXP benchmark seems not to be available, which is a pity since it might be very useful for other 
researchers. I believe the paper may have a stronger impact if the tool would be available. The same is true for the 
data and scripts used for the experiments. I would appreciate if the paper may have an online repository making all 
data and tools available.


\vspace{0.2cm}

{\color{blue}{\bf Answer.} answer}

\vspace{0.2cm}

\item Another point to improve is connected to the implications of the study. While the paper discusses in nice way the 
results achieved, it does not provide a clear view of how those results can be potentially useful for researchers 
working on Android sandboxes as well as what are the next steps that the research community should pursue to 
improve the detection of malwares.


\vspace{0.2cm}

{\color{blue}{\bf Answer.} answer}

\vspace{0.2cm}

\item Last but not least, the threats to validity section should be improved, by commenting on conclusion and construct validity.


\vspace{0.2cm}

{\color{blue}{\bf Answer.} answer}


\end{itemize}

{\bf Reviewer 2}

\begin{itemize}

\item The abstract is not clear. First, the context if not well addressed. For example, why ``mining Android sandboxes'' is 
important? I had to read several related reference at the beginning to understand it. Second, it is not common practice to 
include reference in abstract. However, I reserve my opinion since I know it may be preferred by some researchers.


\vspace{0.2cm}

{\color{blue}{\bf Answer.} answer}

\vspace{0.2cm}

\item In Figure 2, you used ``Malicious'', but in most other places, you used ``Malign''. Please unify your choice of terminology.


\vspace{0.2cm}

{\color{blue}{\bf Answer.} answer}

\vspace{0.2cm}

\item In 3.3, you mentioned ``we investigate whether or not a more advanced static analysis approach is promising for mining 
Android sandboxes. To this end, we leverage the FlowDroid taint analysis algorithms...''. True, FlowDroid has its edge 
in privacy leakage detection, but what's your justification for saying FlowDroid is more advanced in terms of static analysis?


\vspace{0.2cm}

{\color{blue}{\bf Answer.} Thanks for raising this question. Here we were intended to say that FlowDroid implements a more advanced static analysis approach than DroidFax. We agree that the sentence is ambiguous, and we fixed that in the new version of the paper (first paragraph
  of section 3.3).}

\vspace{0.2cm}

\item By definition, mining sandboxes means a technique to confine an application to resources accessed during automatic 
testing [1]. But in 3.3, you are comparing the source-sink pairs between that of a benign app and its malign twin. It is already 
not ``Mining Sandboxes''.

\vspace{0.2cm}

{\color{blue}{\bf Answer.} We totally agree. We changed the title of Section 3.3 and fixed this wrong misconception related to our taint analysis usage, which is used not for mining sandboxes but rather as an alternative approach for identifying malware. Fixing this issue required modifications not only in Section 3.3 but also in many other parts of the paper (including Abstract, Introduction, and Results). Many thanks for raising this issue.}


\vspace{0.2cm}

\item Also, FlowDroid requires as input the sources and sinks APIs. But you mine ``all sources and sinks 
paths''. Do you mean all APIs? FlowDroid will never finish running. 
\vspace{0.2cm}

{\color{blue}{\bf Answer.} You are right: we actually use a set of sensitive sources and sinks. We changed the second paragraph of Section 3.3 to clarify this point. Thanks. We hope that the current version of our paper fixes this misunderstanding.}

\vspace{0.2cm}
  
\item Assume a set of sensitive sources and sinks are selected, 
FlowDroid is already good in suggesting which one is malign by revealing info flows. What's the point of applying [1]'s setting?


\vspace{0.2cm}

{\color{blue}{\bf Answer.} Our idea in the second study is to find divergences in the info flows when considering two executions of FlowDroid: one execution using the benign version and one execution using the malicious version. The source-sink paths collected in the first execution of FlowDroid (benign version)
 {\bf are not considered suspicious}. Only the paths in the second execution, which do not appear in the first execution, are considered suspect and are used to point out the malware. We changed Section 3.3 to detail our approach better.}

\vspace{0.2cm}

{\color{blue}{\bf Answer.} answer}

\vspace{0.2cm}

\item The significance of this work is questionable. The first study is a mere replication of the existing work [2], except, disabling 
part of the framework. The finding is that DroidFaX is important part for DroidXP. Well, your proved literature's correctness. 
It would be more impressive the other way round.


\vspace{0.2cm}

{\color{blue}{\bf Answer.} answer}

\vspace{0.2cm}

\item I find what's written in Finding 2 is not exciting. It is intuitive based on the results of [1] to know that a comprehensive 
comparison combining multiple dynamic tools would improve the performance.


\vspace{0.2cm}

{\color{blue}{\bf Answer.} answer}

\vspace{0.2cm}

\item Listings 1-7 illustrate more details of representative malware in the dataset. But what is the take-away message? What 
conclusions and/or inspirations can be drawn from such listings?


\vspace{0.2cm}

{\color{blue}{\bf Answer.} answer}

\vspace{0.2cm}

\item The title itself is inaccurate. You removed DroidFax from the replicated work[1]. But it is not the only static analysis part. 
Actually, many dynamic testing tools use static analysis to better guide their testing strategy.


\vspace{0.2cm}

{\color{blue}{\bf Answer.} answer}

\vspace{0.2cm}

\item Typos:
1. Page 3: to explains -- to explain.
2. Page 3: BLL-Study study -- BLL-Study.
3. Page 8, does not computed -- does not compute.
4. Page 9, of our experimental -- of our experiments.


\vspace{0.2cm}

{\color{blue}{\bf Answer.} answer}

\end{itemize}

{\bf Reviewer 3}

\begin{itemize}

\item It is not clear how this paper can be considered an extension of the DroidXP paper presented in SCAM 2020.


\vspace{0.2cm}

{\color{blue}{\bf Answer.} answer}

\vspace{0.2cm}

\item Introduction, para 3 is repeating para 2


\vspace{0.2cm}

{\color{blue}{\bf Answer.} answer}

\vspace{0.2cm}

\item ``The authors state that dynamic analysis exceeds static analysis on mining Android sandboxes because the
analysis must often assume that additional behaviors are possible than actually would be.''
Please explain


\vspace{0.2cm}

{\color{blue}{\bf Answer.} answer}

\vspace{0.2cm}

\item ``We presented and evaluated DroidXP in a conference paper [11]''
What is DroidXP? It should be introduced to a reader who is not familiar.


\vspace{0.2cm}

{\color{blue}{\bf Answer.} answer}

\vspace{0.2cm}

\item Google play - Google Play
* java - Java / android - Android
* ``Another approach to rooting''
What is rooting?


\vspace{0.2cm}

{\color{blue}{\bf Answer.} answer}

\vspace{0.2cm}

\item Please rephrase:
``Those are kind of attack that usually tackled with the Android Mining Sandbox''
``For instance, Whaley et al. [18] combine dynamic and static analysis for API mining and so extract interface from software components.''


\vspace{0.2cm}

{\color{blue}{\bf Answer.} answer}

\vspace{0.2cm}

\item What are sensitive behaviors? ``showed that Droidbot is able to trigger an amount of sensitive behaviors''


\vspace{0.2cm}

{\color{blue}{\bf Answer.} answer}

\vspace{0.2cm}

\item Please rephrase:
``From the same authors, another test generator tool for Android, described as Humanoid [23], is a Droidbot evolution and
presents a proposal that can generate humans like tests inputs, using deep learning.''


\vspace{0.2cm}

{\color{blue}{\bf Answer.} answer}

\vspace{0.2cm}

\item Briefly explain the taint analysis
``Taint analysis is a special type of static analysis that enables tracking of sensitive data within programs''


\vspace{0.2cm}

{\color{blue}{\bf Answer.} answer}

\vspace{0.2cm}

\item Please explain:
``FastDroid [28] mitigates this issue by introducing an intermediate light-weight abstraction to perform the analysis.''


\vspace{0.2cm}

{\color{blue}{\bf Answer.} answer}

\vspace{0.2cm}

\item Section 3.1, the help text is not adding much value

\vspace{0.2cm}

{\color{blue}{\bf Answer.} answer}

\vspace{0.2cm}

\item Please rephrase: ``After that, we executed the analysis again, though disabling the DroidFax static analysis algorithm''

\vspace{0.2cm}

{\color{blue}{\bf Answer.} answer}

\vspace{0.2cm}

\item Section 4.1 can be written more concisely.
* It is not clear what is the value of the malware discussion between pages 13 and 15. It may be removed.


\vspace{0.2cm}

{\color{blue}{\bf Answer.} answer}

\vspace{0.2cm}

\item There are several other typos and grammatical errors.


\vspace{0.2cm}

{\color{blue}{\bf Answer.} answer}





\end{itemize}



{\bf Meta review}

We want to thank all comments we received. Based on this feedback, we conducted additional research activities and improved the paper in several directions.

\closing{Yours Faithfully,}


\end{letter}
\end{document}
