\begin{abstract}
The use of sandboxes is an effective technique for malware analysis. However, although the use of dynamic analysis for mining Android sandboxes has been investigated before, little is known about the potential benefits of combining static and dynamic analysis for mining Android sandboxes.
Accordingly, in this paper we present the results of two studies that investigate whether or not static analysis might complement and increase the performance of dynamic analysis tools for mining Android sandboxes. %\emph{Method.} 
In the first study we conduct a non-exact replication of the Bao et al. work, a previous study that compares the performance of test case generation tools for mining Android sandboxes. Differently from the original work, here we isolate the effect of the static analysis tool (DroidFax) they used to instrument the Android apps in their experiments. This decision was motivated by the fact that DroidFax could have influenced the efficacy of the dynamic analyses positively---through the execution of specific static analysis algorithms. In our second study, we carried out a new experiment to investigate the efficacy of taint analysis algorithms to complement mining Android sandboxes. To this end, we executed the FlowDroid tool to mine the source-sink flows from the benign/malign pairs of Android apps used in the Bao et al. study.%\emph{Results.}
Our study brings several findings. For instance, the first study reveals that DroidFax alone (static analysis) can detect 43.75\% of the malwares in the dataset of the Bao et al. work, leading to an overestimation of the performance of the dynamic analysis tools. The results of the second study show that (static) tainted analysis is also practical to complement mining Android sandboxes approach, with a performance similar to that reached by dynamic analysis tools.
\end{abstract}

\begin{keyword}
Malware Detection \sep Mining Sandboxes \sep Android Platform \sep Static Analysis and Dynamic Analysis 
\end{keyword}
