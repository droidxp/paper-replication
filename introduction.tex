\section{Introduction}\label{sec:introduction}

Almost two-thirds of the world use mobile technologies~\cite{Comscore}, and the Android Operating System has dominated the market of smartphones, tablets, and others electronic devices \cite{statcounter}. Due to this growing popularity, the number of incidents related to Android malicious software (malware) has significantly increased. Malware samples grow from just three families and 100 samples in 2010 to more than hundred families with thousands of samples in 2013~\cite{DBLP:journals/comsur/FarukiBLGGCR15,DBLP:journals/csur/SufatrioTCT15}%\kn{Can we here say by how much?}
. Security issues in Android software applications~\footnote{In this paper, we will use the terms Android Applications, Android Apps and Apps interchangeably, to represent Android software applications} have become a relevant research topic, and many techniques have been developed to identify vulnerabilities in Android apps~\cite{DBLP:conf/pldi/ArztRFBBKTOM14}, including the use of static analysis algorithms either to identify privacy leaks or to reveal the misuse of cryptographic primitives~\cite{krueger:ecoop-2018,rahaman:ccs-2019}, for instance.

Another perspective to the solving the same problem are approaches that rely on the use of dynamic analysis to mine Android sandboxes. Jamrozik et al. proposed BOXMATE tool~\cite{DBLP:conf/icse/JamrozikSZ16}, which with dynamic approaches, during an exploratory phase, one leverages an automatic test case generator tool that runs a benign version of an Android app and collects the set of calls made to sensitive APIs
%\kn{Is this the case for just Jamrozik's work or all of dynamic approaches. It is not clear}
. This set of calls comprises a sandbox infrastructure. The sandbox might then monitor any call to sensitive APIs while the app is running, blocking calls that have not been identified during the exploratory phase---thereby protecting Android users from additional malicious behavior~\cite{DBLP:conf/icse/JamrozikSZ16}.
Jamrozik et al. argue in favor of dynamic analysis for mining sandboxes, instead of using static analysis---mostly because of the overapproximation problem: ``static analysis often assume that more behaviors are possible than actually would be''~\cite{DBLP:conf/icse/JamrozikSZ16}. In addition, code that uses dynamic features (such as reflection) poses additional challenges to static analysis algorithms---even though \emph{dynamic features} of programming languages are often used to introduce malicious behavior. Even though these claims are reasonable, previous research results do not present empirical assessments about the limitations of static analysis to mine sandboxes. Consequently, it is not clear whether and how both approaches (dynamic and static analysis) could complement each other in the process of mining Android sandboxes.

The same lack of understanding about static and dynamic analysis complementing each other also appears in the work of Bao et al.~\cite{DBLP:conf/wcre/BaoLL18} (hereafter \blls), which presents an empirical study that compares different dynamic analysis tools for testing and mining Android sandboxes%\kn{The use of word same here makes me wonder if readers could be confused. IMO the problem of how Static and dynamic analysis complement each other AND the role of static analysis in dynamic approaches are different (even if they sound similar), right? Reason I simply did not remove the word "same" is that I feel we need a bridge sentence here to make this transition of two similar sounding problems clear, because we only explore one of them.}
. Their study leverages DroidFax~\cite{DBLP:conf/icsm/CaiR17a} to instrument $102$ pairs of Android apps (each pair comprising a benign and a malicious version of an App) and to collect the information needed to mine sandboxes (that is, the calls to sensitive APIs).
Although the authors report a precision of at most 70\% of dynamic analysis tools to differentiate the benign and malicious versions of the apps, the authors ignore the fact that DroidFax statically analyzes the Android apps and also records calls to sensitive APIs (besides instrumenting the apps), which, as we discuss in this paper, leads to an overestimation of the performance of the dynamic analysis tools for mining sandboxes and might have introduced a possible threat to the conclusions of that work. Moreover, since DroidFax is an independent component (it is not part of the dynamic analysis tools explored in the \blls), a practitioner might run a dynamic analysis tool alone that is likely to present a smaller capacity to identify malware. 

The goal of this paper is two fold. First we present the results of an
external, non-exact replication~\cite{role-of-replication} of the \blls. To this end,
we take advantage of DroidXP, a tool suite that helps researchers (including ourselves) to
integrate test case generation tools and compare their performance on
mining Android sandboxes. We discussed the design and implementation of DroidXP in a conference
paper~\cite{DBLP:conf/scam/CostaMCMVBC20}, which also
includes an initial evaluation of DroidXP.
As a matter of fact, the results of the first DroidXP evaluation revealed a possible
overestimation in the performance of dynamic analysis tools as
reported in the \blls---which in the end motivated us to
conduct the non-exact replication of that study. Here we extend
our previous work with a slight customization of DroidXP, which allowed us
to reproduce the \blls by means of a serie of experiments
that reveal the actual performance of the
dynamic analysis tools. Section~\ref{sec:droidxp} revisit the
DroidXP design, while Section~\ref{sec:set1} discuss
the setup of our replication study.


%% In summary, our current study differs fromthe original work in three aspects: 

%% \begin{enumerate}[(a)]
%%   \item We isolate the effect of the static analysis component of DroidFax.
%%    \item We included a fake test
%% case generation tool (named \joke). \joke simulates a test tool that does not execute Android apps during DroidXP execution, disregarding the effects of dynamic analysis, and computing the results with only static analysis DroidFax component.
%% \item We extend the execution time and the number of executions of each test case generator tool for all apps
%% in our dataset.

%% \end{enumerate}
%% where offers direct access to all source code, installation process, dependencies, and how to run. We presented and evaluated DroidXP in a conference paper~\cite{DBLP:conf/scam/CostaMCMVBC20}, and here we discuss the results of a broader investigation that also uses DroidXP to explain the performance of static and dynamic analysis approaches for mining sandboxes. Our goal is to understand how static analysis algorithms might complement dynamic analysis approaches in the process of mining Android sandboxes. This work is an extension of SCAM 2020 paper, since that for our actual study, we had to adapt DroiXP to allow it to isolate the effects of static analysis component of DroidFax, and take into account this effect at our analysis.

%% Our research work presents the results of two empirical studies that aim to
%% address our research goal. In the
%% first study, we conduct a replication of the \blls, as presented at {\color{red}SCAN 2020} conference~\cite{DBLP:conf/scam/CostaMCMVBC20}, though 

Second, in this paper we also explore how a static analysis approach
(based on taint analysis) compares and complements the mining sandbox technique
for identifying malicious behavior that infects benign applications.
The idea here is to compare the dataflows from \emph{source} to
\emph{sink} statements computed using two executions of the
FlowDroid infrastructure: one execution that analyses
a benign version of an Android app and one execution that
analyses a malicious version. We consider that
the taint analysis approach is able to identify a malware whenever
we find a dataflow from a source to a sink in the second execution
that does not appear in the first one. We detail the
settings of this taint analysis study in Section~\ref{sec:set2}

Altogether, this paper brings the following contributions:

\begin{itemize}
\item A replication of the \blls that better clarifies the performance of
  dynamic analysis tools for mining Android sandboxes. The results of
  our replication (Section~\ref{sec:res-fs})
  give evidence that the previous work overestimated
  the performance of the dynamic analysis tools---that is, without
  DroidFax (an independent component used for running the
  \blls experiment), the performance of the tools drop between $16.44$\% to $58$\%. 

\item A broad comprehension about the role of static analysis tools for mining
  sandboxes, showing that we can benefit from using both static and dynamic
  analysis for detecting malicious Android apps. In addition,
  we give evidence that a well known static analysis approach, based on
  taint analysis, leads to a performance similar to the dynamic analysis
  approach for diferenciating benign and malicious versions of the same
  app (Section~\ref{sec:res-ss}).

\item A reproduction package of our study that is available online, including
  scripts for statistic analysis \footnote{https://htmlpreview.github.io/?https://github.com/droidxp/paper-replication-package/blob/master/replication.html}
  and a benchmark for reproducing and extending our study. The repository for DroidXP is available
at GitHub\footnote{https://github.com/droidxp/benchmark}.
 
\end{itemize}
