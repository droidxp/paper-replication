\section{Implications}\label{sec:implications}

The results discussed so far bring evidence
that the \blls study overestimated the
performance of the dynamic analysis tools in
malware identification using the mining sandboxes.
This finding has implications for both
researchers and practitioners. First,
we revisit the literature showing that
DroidFax alone is also effective for mining
sandboxes, being able to identify 43.75\%
of the malwares in our dataset. Moreover,
DroidFax identifies malwares that
none of the generated sandboxes
were able to find, increasing the
performance of the sandbox in at most $51.79\%$ (in the
case of Humanoid).

Table~\ref{tab:fs} in the previous section summarizes this finding: when
executing the mining sandbox approach without the
support of DroidFax static analysis, Humanoid's sandbox
could identify only 27 malwares ($28.12\%$ of the
malwares in our dataset).
Conversely, the DroidBot sandbox achieved the best performance in
terms of the number of detected malware without the DroidFax support for static analysis,
being able to identify $63.54\%$ of the malwares.
The message
here is that researchers and practitioners should
explore the use of DroidFax (or a similar tool) in conjunction 
with dynamic analysis techniques for mining sandboxes---
reviewing the findings of the \blls~\cite{DBLP:conf/wcre/BaoLL18}
and enriching the discussion about
the limitations of static analysis for
mining sandboxes~\cite{DBLP:conf/icse/JamrozikSZ16}.

In the second study we used FlowDroid
to explore a novel approach for malware identification,
which aims to compare the source-sink paths of two
versions of an app (one known to
be secure and another that might
have been repackage or that might have
an injected malicious behavior).
Contrasting with the static
analysis limitations discussed in~\cite{DBLP:conf/icse/JamrozikSZ16},
our findings indicate that
this approach is also effective for malware
identification. Indeed, our taint
analysis approach using FlowDroid
detects several malwares
that none of the sandboxes generated
with the dynamic analysis tools (plus
the DroidFax static analysis component)
could identify (see Table~\ref{tab:taint}). These
result has also implications
for both academia and industry. First,
this it reinforces the benefits
of integrating both static and
dynamic analysis for malware identification.
Second, this finding suggests that
practitioners can benefit from using
an integrated approach that
combines the mining sandbox approach
with taint analysis for malware
identification. 

\begin{table}[ht]
\centering
\begin{tabular}{lccc}\toprule
 Test Generation & FlowDroid & Total & \%\\
 Tool & Increase  &  & \\ \midrule
 DroidBot & 6 & 79 & 82.29\\
 Monkey & 7 &  78 & 81.25 \\
 DroidMate & 7 & 75 & 78.12  \\
 Humanoid & 16 & 72 & 75.00 \\
 Joker & 25 & 67 & 69.79  \\\midrule
 
\end{tabular} 
\caption{Malwares detected in 96 pair (B/M) increased by the taint analysis approach}
\label{tab:taint}
\end{table}

% research community

% practitioners

%%here we also have to write about the intersection between droidfax and tainted analysis now, and about what 1 reveled and other dont, and otherwise. 


%Here we can write about MOTIVATION EXAMPLES.
